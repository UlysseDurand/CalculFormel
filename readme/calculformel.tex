\documentclass{article}
%\usepackage{tikz}
\usepackage[margin=1in]{geometry} % full-width
%\usepackage{graphicx}

%NOTES POUR M.CANONICO : Cours sur les automates de Gleize
%http://gleize.eu/optioninfo/cours/chap4_automates.pdf
%prerequis : http://gleize.eu/optioninfo/cours/chap3_langages.pdf
%en OCAML : -> correspond à celui en mathématiques sur les ensembles.
%         : si l est une liste et e un élément, (e::l) est la liste avec l à laquelle on a ajouté l'élément e
%         : si l est m sont deux listes, l@m est la concaténation des deux listes
%         : les polymorphismes en OCAML, c'est les types 'q. C'est un type générique, pouvant être remplacé par la suite.
%         ! la currification en OCAML, une fonction 'a->'b->'c, par exemple l'addition, int -> int -> int = int -> (int -> int). à un entier x on associe une fonction du type (int->int) qui à un entier y associe x+y.


% AMS Packages
\usepackage{amsmath}
\usepackage{amsthm}
\usepackage{amssymb}

\usepackage[T1]{fontenc}
\usepackage{tikz}

\usepackage{fancyvrb}

% Unicode
\usepackage[utf8]{inputenc}


\theoremstyle{definition}
\newtheorem{definition}{Definition}[section]

\newtheorem{implementation}{Imp\'ementation OCaml}[section]
\newcommand{\warncharlist}{{\color{blue}(On confond \codecml{string} et \codecml{char list})}}
\newcommand*{\unefiguregraphic}[2]{\begin{figure}[!ht] \centering \includegraphics{#1} \caption{#2} \end{figure}}
\newcommand*{\unefiguregraphicsimple}[1]{\begin{figure}[!ht] \centering \includegraphics{#1}\end{figure}}
\newcommand*{\unefiguretikz}[2]{\begin{figure}[!ht] \centering \input{figures/#1.tikz} \caption{#2} \end{figure}}
\newcommand*{\unefiguretikzsimple}[1]{\begin{figure}[!ht] \centering \input{figures/#1.tikz}\end{figure}}

\newcommand{\chcar}[1]{{\fontfamily{lmtt}\selectfont \color{codepurple}\textit{#1}}}
\newcommand{\codecml}[1]{\colorbox{backcolour}{\fontfamily{phv}\selectfont #1}}
\usepackage{xcolor}

\definecolor{codegreen}{rgb}{0,0.6,0}
\definecolor{codegray}{rgb}{0.5,0.5,0.5}
\definecolor{codepurple}{rgb}{0.58,0,0.82}
\definecolor{backcolour}{rgb}{0.95,0.95,0.92}
\usepackage{listings}
\usepackage{cprotect}
\lstdefinestyle{mystyle}{
    backgroundcolor=\color{backcolour},
    commentstyle=\color{codegreen},
    keywordstyle=\color{magenta},
    numberstyle=\tiny\color{codegray},
    stringstyle=\color{codepurple},
    basicstyle=\ttfamily\footnotesize,
    breakatwhitespace=false,
    breaklines=true,
    captionpos=b,
    keepspaces=true,
    %numbers=left,
    numbersep=5pt,
    showspaces=false,
    showstringspaces=false,
    showtabs=false,
    tabsize=2
}
\lstset{
    language=caml,
    style=mystyle,
    breaklines=true
}

\title{Calcul formel}
\author{DURAND Ulysse}
\everymath{\displaystyle}
\date{}
\begin{document}

\maketitle
Les expressions ou morceaux de code OCaml seront \codecml{sous cette forme}.\\
Les cha\^ines de caract\`eres seront \chcar{sous cette forme}.\\\\
Le but est de faire ce qui peut s'apparenter \`a un logiciel de calcul formel en OCaml, il doit pouvoir \'evaluer des expressions, les d\'eriver, et pouvoir les afficher en \LaTeX.\\
Un objectif est aussi de pouvoir renseigner au programme les expressions des fonctions \`a traiter en \LaTeX.\\
En compl\'ement, le programme peut aussi partiellement simplifier des expressions. Ansi on peut simplifier $(0+1)*x_0$ par $x_0$ \\
On aura \`a la fin, cr\'e\'e ces fonctions :
\begin{lstlisting}
val evalue : expression -> float array -> float = <fun>
val affiche : expression -> string = <fun>
val derive : int -> expression -> expression = <fun>
val latex_vers_expression : string -> expression = <fun>
val simplifie : expression -> bool * expression = <fun>
\end{lstlisting}
(\codecml{simplifie} retourne aussi un booléen, il sera à \codecml{true} si l'expression donnée est simplifiable et \codecml{false} sinon.)\\\\
Voci un exemple de ce dont le programme est capable :
\\
Le code suivant :

\begin{lstlisting}
let exprlatexa = "x_0^{x_1}+x_1^{x_0}";;

let expra = latex_vers_expression exprlatexa;;
(*parse le latex*)

let exprb = (derive 0 expra);;
(*derive l'expression par rapport a x_0*)

let exprbsimpl = snd (simplifie exprb);;
(*simplifie l'expression obtenue*)

let exprlatexb = affiche exprbsimpl;;
(*transforme l'expression en latex*)

print_string (exprlatex b);;
\end{lstlisting}
Retournera :\\
\chcar{(frac\{\{x\_\{1\}\}\}\{\{x\_\{0\}\}\})*(\{x\_\{0\}\} \{x\_\{1\}\})+(ln(\{x\_\{1\}\}))*(\{x\_\{1\}\}\{x\_\{0\}\})}\\
qui s'affiche ainsi en latex : \\
$\displaystyle \left(\frac{{x_{1}}}{{x_{0}}}\right)*({x_{0}}^{x_{1}})+(ln({x_{1}}))*({x_{1}}^{x_{0}})$\\
Et il s'agit bien de la d\'eriv\'ee par rapport \`a $x_0$ de $x_0^{x_1}+x_1^{x_0}$.\\\\

Remarque : Sans la simplification de l'expression d\'eriv\'ee, on aurait ce r\'esultat, juste mais peu accueillant.\\
$((0.)*(\ln({x_{0}}))+({x_{1}})*((1.)*(\frac{1}{{x_{0}}})))*(exp(({x_{1}})*(\ln({x_{0}}))))+((1.)*(\ln({x_{1}}))+({x_{0}})*((0.)*(\frac{1}{{x_{1}}})))*(exp(({x_{0}})*(\ln({x_{1}}))))
$

\section {Repr\'esentation des fonctions}
Nous traiterons les expressions alg\'ebriques en les repr\'esentant par des arbres d'expression.\\

l'arbre Figure1 repr\'esente l'expression alg\'ebrique $\frac{x+3}{-y}$ : \\
\unefiguretikz{fig1}{Exemple d'arbre d'expression}
\\
On discernera :
\begin{itemize}
    \item Les feuilles qui sont des variables ou des constantes
    \item Les n\oe{}uds qui sont de la forme (une op\'eration, des fils)
\end{itemize}
d'o\`u une telle impl\'ementation en OCaml :

\begin{lstlisting}
type operation = {nbvar : int ;affichage : string ; evaluation : float array->float ; derive : deriv}
and expression =
    C of float |
    V of int |
    F of operation * (expression array)
and deriv = NonDeriv | Deriv of ((expression array)->(expression -> expression) -> expression);;
\end{lstlisting}

Les variables sont caract\'eris\'ees par un entier ($x_i, i\in[|0,n-1|]$) o\`u $n$ est nbvar, le nombre de variable en entr\'ee de l'op\'eration (son nombre de fils).\\
Nous allons d\'etailler les caract\'eristiques \codecml{affichage}, \codecml{evaluation} et \codecml{derive} des op\'erations et le type \codecml{deriv}.\\



\section{affichage}
Construisons une fonction \codecml{affiche}: \codecml{expression -> string}, telle que si a est l'arbre d'expression Figure1, alors \codecml{affiche a} retourne \chcar{frac\{x+3\}\{-y\}}\\
Pour expliciter comment afficher une expression, nous allons, \`a chaque op\'eration, associer un code dans un language invent\'e (nous allons l'appeler LanguInv), ce code \'etant sous forme d'une cha\^ine de caract\`eres, il d\'ecrit la forme de l'affichage d'une op\'eration.

\subsection{le language LanguInv}
Ce language invent\'e a pour but d'exprimer des motifs. C'est \`a dire, une forme de cha\^ine de caract\`ere, avec ce qui s'appelle des placeholder, c'est \`a dire des emplacements libres. Par la suite, soit on replacera ces placeholder par des cha\^ines de caract\`eres, par exemple pour afficher du \LaTeX (dans cette partie), soit on reconna\^itra le motif et alors on retournera les cha\^ines de caract\`eres \`a la place des placeholders (dans le parser).\\\\
L'usage le plus courant qu'on aura, c'est celui o\`u on renseigne un seul placeholder \`a la fois. Alors la syntaxe pour d\'esigner le placeholder $i$ sera \chcar{\%|}i\chcar{;\%}.\\
On a d\'ej\`a de quoi construire le motif latex de l'op\'eration d'addition : \chcar{\%|0;\% + \%|1;\%}.\\
Alors on veut par exemple que notre fonction \codecml{affiche}, appliqu\'ee \`a \codecml{(F(plus,[|a ; b|]))} o\`u \codecml{a} et \codecml{b} sont du type \codecml{expression}, retourne \codecml{(affiche a)}\chcar{ + }\codecml{(affiche b)}.\\
Notre fonction aura remplac\'e dans le motif du plus, le placeholder 0 par l'affichage du fils 0, soit \codecml{a}, et le placeholder 1 par l'affichage du fils 1, soit \codecml{b}. \\\\

Une fonctionnalit\'e plus avanc\'ee du language est celle de pouvoir renseigner plusieurs placeholder \`a la fois, s\'epar\'es par une cha\^ine de caract\`eres appel\'ee d\'elimiteur. La syntaxe est la suivante : \chcar{\%}delimiteur\chcar{|}i\chcar{-}j\chcar{;\%} pour d\'esigner les placeholder de i \`a j-1, s\'epar\'es par le delimiteur.\\
Ainsi, les deux expressions en LanguInv suivantes correspondent au m\^eme motif : \\
\chcar{determinant[\%;,;|0-4;\%]}\\
\chcar{determinant[\%|0;\%;,;\%|1;\%;,;\%|2;\%;,;\%|3;\%]}\\
\\
Une toute derni\`ere fonctionnalit\'ee : si j n'est pas renseign\'e, alors il a pour valeur de base \codecml{ope.nbvar}, et si i n'est pas renseign\'e, alors il a pour valeur de base 0.

\subsection{Interpreteur de ce language}
Comment faire comprend \`a OCaml ce language ?\\
Nous allons utiliser un type d'automate particulier qui nous permettra de faire l'automate \codecml{(autospecial n rendph s)} qui reconna\^it un tel language,
ce type particulier d'automate sera appel\'e ici automate d\'eterministe qui \'ecrit avec m\'emoire.\\
Il \'ecrit,
c'est \`a dire qu'il reconna\^it des mots mais renvoit aussi une cha\^ine de caract\`eres.\\
On peut donc associer \`a un tel automate une fonction (ce sera $\eta^\star((i,mem_i),\cdot)$, explications dans la section 3). \\On fera la correspondance par une fonction \codecml{fonction\_d\_auto : automate\_quiecrit -> (char list -> char list)}. \warncharlist

L'automate \codecml{autospecial n rendph s} sera associ\'e \`a une fonction \codecml{evaluelatex = fonction\_d\_auto autospecial}.\\
\codecml{autospecial} d\'epend des param\'etres suivants :
\begin{itemize}
    \item \codecml{ope.nbvar} le nombre de variables de l'op\'eration
    \item une fonction \codecml{rendph} (rendu placeholder), qui elle, prend un entier i en entr\'ee et qui lui associe par quoi il faut remplacer le placeholder i, ici, \codecml{(rendph i)} retournera simplement \chcar{x\_}i
    \item \codecml{ope.affichage} : notre code en LanguInv pour le motif de l'op\'eration
\end{itemize}
\codecml{evaluelatex} retournera l'affichage voulu pour notre op\'eration.\\
Voici le typage de \codecml{evaluelatex} : \codecml{int -> (int->char list) -> char list -> char list}\\
Par exemple, \codecml{evaluelatex 3 string\_of\_int "ope(\%;|-;\%)"} retournera :
\chcar{ope(0;1;2)} \warncharlist\\
Les d\'etails de l'automate et de sa fonction, \codecml{evaluelatex}, associ\'ee, seront explicit\'es dans la section 3.2.

\section{De quoi construire la fonction \codecml{evaluelatex}}
\subsection{Automates utiles}
En OCaml, voici une manière d'implémenter les automates :

\begin{lstlisting}
type ('q, 's) automate_nondet = ('q list)*('q -> bool)*('q*'s*('q list));;
type ('q, 's) automate_det = ('q)*('q -> bool)*('q*'s*'q);;
(*('q,'s) automate_* = (i,f,delta) correspond a un automate ('q,'s,i,f,delta)*)
(*'q : ensembles des etats, 's : alphabet des mots a reconna\^itre, i : etats initiaux, f : etats finaux, delta : fonction de transition*)
\end{lstlisting}



\subsubsection{Automate déterministe qui écrit}
\begin{definition}[Automate déterministe qui écrit]
Ce type d'automate permet de transformer un mot en un autre.\\

\begin{gather*}
    \mathcal{A} = (Q,\Sigma_1 ,\Sigma_2, i, F, \delta, \eta) \\
\end{gather*}

$Q$ est l'ensemble des \'etats, comme dans un automate standard.\\
$\Sigma_1$ est l'alphabet d'entr\'ee, comme dans un automate standard.\\
$\Sigma_2$ est l'alphabet de sortie, c'est dans cet alphabet que l'automate \'ecrira.\\
$\delta$ est la fonction de transition, comme dans un automate standard.\\
$\eta$ est la fonction d'\'ecriture, ressemblant \`a $\delta$, \`a un \'etat $q\in Q$ et une lettre $l \in \Sigma_1$, elle associera un mot \`a \'ecrire $m \in (\Sigma_2)^\star$.

\begin{gather*}
    i\in Q,F \in \mathcal{P} (Q)
\end{gather*}
\begin{align*}
    \delta : Q\times \Sigma_1 &\rightarrow Q \\
    \eta : Q\times \Sigma_1 &\rightarrow (\Sigma_2 )^\star\\
    \\
    \delta^\star : Q\times \Sigma_1 ^\star &\rightarrow Q \\
    (q,\epsilon) &\mapsto q\\
    (q,l.m) &\mapsto \delta^\star(\delta(q,l),m)\text{, avec }l\in \Sigma_1\\
    \\
    \eta^\star : Q\times \Sigma_1 ^\star &\rightarrow (\Sigma_2)^\star \\
    (q,\epsilon) &\mapsto \epsilon '\\
    (q,l.m) &\mapsto \eta(q,l).\eta^\star(\delta(q,l),m)\text{, avec }l\in \Sigma_1\\
\end{align*}

\end{definition}
Impl\'ementation en OCaml :
\begin{lstlisting}
type ('q, 's, 't) automatequiecrit = ('q*('t list), 's ) automate;;
\end{lstlisting}
\codecml{('q,'sig1,'sig2) automatequiecrit = (i,f,g)} correspond \`a un automate $('q,'s,'t,i,f,delta,eta)$. \\
Si $delta(q,l)=q'$ et que $eta(q,l)=p$, alors $g$ est telle que $g((q,m),l) = (q',p::m)$.
\paragraph{correspondance entre automate qui \'ecrit et fonction de $(\Sigma_1)^\star$ dans $(\Sigma_2)^\star$ }:\\
\`Le but de l'automate qui \'ecrit est de cr\'eer une fonction de $(\Sigma_1)^\star$ dans $(\Sigma_2)^\star$.\\
Pour l'automate $\mathcal{A} = (Q,\Sigma_1 ,\Sigma_2, i, F, \delta, \eta)$, cette fonction est la fonction
\begin{align*}
    (\Sigma_1)^\star &\rightarrow (\Sigma_2)^\star\\
    m &\mapsto \eta^\star(i,m)
\end{align*}
\subsubsection{Automate avec mémoire}
Ce type d'automate n'a pas de d'intérêt théorique, mais il permet de mieux se repr\'esenter des automates un peu compliqu\'es.
\begin{definition}[Automate qui écrit avec mémoire] :\\
C'est un automate standard o\`u pour ensemble d'\'etats on prend $(Q \times S_m)$. \\Pour un \'etat $(q,m)$ d'un tel automate, on appelera $q\in Q$ l'\'etat pur et $m\in S_m$ l'\'etat m\'emoire.
\end{definition}
Ainsi, on se repr\'esente toujours les automates avec les \'etats de Q mais maintenant, avec un \'etat de m\'emoire associ\'e.\\
Maintenant pour un automate avec m\'emoire $\mathcal{A} = (Q\times S_m,\Sigma, i, F, \delta)$,\\
$i\in (Q\times S_m)$,\\
$F \in \mathcal{P}(Q\times S_m)$,\\
$\delta : (Q\times \S_m) \times \Sigma \rightarrow (Q\times S_m)$.
\subsubsection{Automate qui \'ecrit avec m\'emoire}
On peut maintenant simplement combiner la d\'efinitions d'un automate qui \'ecrit et celle d'un automate avec m\'emoire pour avoir celle d'un automate qui \'ecrit avec m\'emoire.\\
Un tel automate sera sous cette forme : $\mathcal{A} = (Q\times S_m,\Sigma_1 ,\Sigma_2, i, F, \delta, \eta)$\\
On aura pour cet automate les fonctions $\delta^\star$ et $\eta^\star$.\\
Impl\'ementation en OCaml : \\

\begin{lstlisting}
type ('q, 'm, 's, 't) automate_quiecrit_avecmemoire = ('q*'m,'s,'t) automatequiecrit
\end{lstlisting}

Pour les représenter graphiquement, nous utiliserons, comme les automates, un graphe orient\'e, mais avec des annotation sur les arêtes différentes :\\

\unefiguretikzsimple{lareprstand}
(
L'indice 0 d\'esigne le premier \'element du couple qu'est l'\'etat $\delta((a,mem),l)$, c'est l'\'etat pur. l'indice 1 est pour d\'esigner le deuxi\`eme \'element de ce m\^eme couple, c'est l'\'etat m\'emoire.)

\subsection{Et dans notre cas}
Voici l'automate \codecml{autospecial n rendph s} (correspond \`a $\delta^\star(i,mem_i,s)$ pour l'automate suivant):\\
avec $n$ un entier et $N = \{\chcar{0},\chcar{1},
\chcar{2},\chcar{3},\chcar{4},\chcar{5},\chcar{6},\chcar{7},\chcar{8},\chcar{9}\}$\\

\unefiguretikz{fig2}{Automate de \codecml{evaluelatex}}
Notre automate qui \'ecrit avec m\'emoire est d\'eterministe, il a pour ensemble d'\'etats pur $\{0,1,2,3\}$, pour ensemble d'\'etats m\'emoire \codecml{(char list)*int*int}, le \codecml{char list} sert \`a stocker le d\'elimiteur en m\'emoire, et \codecml{int*int} \`a stocker le i et j d\'esignants les bornes de l'ensemble des indices de placeholder. L'automate a omme \'etat initial $0$, comme ensemble d'\'etats finaux $\{0\}$ et comme alphabet de d\'epart et d'arriv\'ee, \codecml{char}.\\
Si A est cet automate qui écrit, \codecml{n = 6}, \codecml{rendph a} retourne \chcar{x\_}a (a entier)\\
fung est d\'efini ainsi en OCaml :
\begin{lstlisting}
let rec fung (s,a,b) =
    if a > b then [] else
    if a = b then rendph a
    else (rendph a)@s@(fung (s,a+1,b)) in
\end{lstlisting}
Par exemple, \codecml{fung (s,0,3) = (rendph 0)\^{}s\^{}(rendph 1)\^{}s\^{}(rendph 2)\^{}s\^{}(rendph 3)}\\\\
On arrive \`a ce r\'esultat, satisfaisant : \\
$\eta^\star ( (0,(\text{''},0,0)),$
\chcar{frac\{\{\%|0;\%+\%|15;\%\}\^{}\{(\%*|1-4;\%)\}\}\{\%+|a;\%\}}) =\\
\chcar{frac\{\{x\_0+x\_\{15\}\}\^{}\{x\_1*x\_2*x\_3\}\}\{x\_0+x\_1+x\_2+x\_3+x\_4+x\_5\}}\\
\paragraph{Explications partielles}:\\
Sur l'\'etat 0, l'automate recopie ce qui rentre tant qu'on ne tente pas de mettre un placeholder.\\
Quand on veut mettre un placeholder, c'est quand on reconna\^it le \chcar{\%}, alors on passe dans la partie de traitement du placeholder, soit les \'etats $1$, $2$, $3$.\\
Dans cette partie de traitement du placeholder, on n'\'ecrit que lorsqu'on revient \`a l'\'etat 0 (en reconna\^issant un autre \chcar{\%}), alors on \'ecrit ce que l'on veut \`a la place du placeholder, via les fonctions \codecml{rendph} et \codecml{fung}.\\
Dans cette partie, l'ensemble des transitions internes permettent de changer la m\'emoire en interpr\'etant ce qui \'etait voulu pour d\'elimiteur et pour valeurs de i et j.


\subsubsection{Resultat}
Maintenant qu'on a la fonction \codecml{evaluelatex}, on peut enfin construire notre fonction \codecml{affiche} :
\begin{lstlisting}
let rec affiche f = match f with
    |C(x) -> string_of_float x
    |V(i) -> "{x_{"^(string_of_int i)^"}}"
    |F(g,va) -> evaluelatex (g.nbvar) (fun a -> List.rev (explode (affiche va.(a)))) (g.affichage);;
\end{lstlisting}

\section{Evaluation}
Construisions une fonction evalue qui prend en entr\'ee une expression et un vecteur en lequel l'\'evaluer, donc du type : \codecml{expression -> float array -> float}, telle que si a est l'arbre d'expression Figure1, alors \codecml{evalue a [|2.;3.|]} retourne \codecml{-1.6666}\\
L'\'evaluation est alors plut\^ot simple, se faisant par induction.\\
\begin{itemize}
\item Pour une constante, on retourne la constante
\item Pour la variable $x_i$ on retourne \codecml{v.(i)}
\item Pour une op\'eration sur plusieurs expressions, on retourne la fonction d'\'evaluation de l'op\'eration appliqu\'ee \`a l'\'evaluation de chaque fils.
\end{itemize}

d'o\`u une telle impl\'ementation en OCaml :
\begin{lstlisting}
let rec evalue f v = match f with
    |C(x) -> x
    |V(i) -> v.(i)
    |F(g,fa) -> g.evaluation (Array.map (fun unef -> evalue unef v) fa );;
\end{lstlisting}

\section{D\'erivation}
Pour renseigner pour une op\'eration si elle est d\'erivable et comment \'evaluer sa d\'eriv\'ee, nous allons utiliser le type d\'erivation d\'efini en premi\`ere page que nous rappelons : \\
\codecml{deriv = NonDeriv | Deriv  of (( expression  array)->(expression  -> expression) ->expression)}.\\
Soit l'op\'eration n'est pas d\'erivable, soit elle l'est et alors on donne une expression de sa d\'eriv\'ee.\\
Pour l'expression de sa d\'eriv\'ee nous aurons besoin d'utiliser la fonction \codecml{derive} \`a l'int\'erieur, qui sera alors renseign\'ee dans Deriv (c'est le \codecml{expression -> expression}). Le \codecml{expression array} correspond aux fils de la fonction qui auront leur r\^ole dans l'expression de la d\'eriv\'ee.\\
Alors on a \codecml{Derive} prenant en param\`etres \codecml{ar} et \codecml{d}, \codecml{ar} \'etant les fils de l'op\'eration et \codecml{d} \'etant la fonction de d\'erivation.\\\\

Un exemple sur l'op\'eration plus,\\
\codecml{plus.derive = Deriv(fun ar d -> F(plus,[|d ar.(0);d ar.(1)|]) )}.\\
On retrouve $(u+v)'=u'+v'$\\
Un deuxi\`eme exemple, sur l'op\'eration de multiplication,
\\ \codecml{fois.derive = Deriv(fun ar d -> F(plus,[| F(fois,[|d ar.(0);ar.(1)|]) ; F(fois,[|ar.(0);d ar.(1)|]) |]))}.\\
On retrouve $(u*v)' = u'*v + u*v'$\\\\

Pour ce qui est de la fonction \codecml{derive}, il y a plusieurs fonctions de d\'erivation que l'on pourra renseigner dans le type Derive de notre op\'eration, ce sont les d\'eriv\'ees par rapport aux diff\'erentes variables. \\C'est pourquoi nous allons prendre en arguments un entier \codecml{k} et une expression \codecml{f} pour retourner l'expression de la d\'eriv\'ee selon la variable $x_k$.\\
Si \codecml{f} est la variable $x_i$, alors on retourne $\delta_{k,i}$.\\
Si \codecml{f} est une constante, alors on retourne le flottant \codecml{0}.\\
Si \codecml{f} est une op\'eration \codecml{g} dont les fils sont \codecml{va} et que \codecml{g.derive} est \codecml{NonDeriv}, on renvoit une erreur mais si c'est \codecml{Deriv(laf)} alors tout est renseign\'e dans \codecml{laf}, il n'y a qu'a retourner \codecml{laf va (derive k)}\\
Voici l'expression de \codecml{derive} en OCaml : \\
\begin{lstlisting}
let rec derive k f = match f with
	|C(x)->C(0.);
	|V(i)-> if i=k then C(1.) else C(0.);
	|F(g,va) -> match g.derive with
		|NonDeriv->failwith "Fonction non derivable !";
		|Deriv(laf)->laf va (derive k);;
\end{lstlisting}
\section{Parser}
Le plus dur reste \`a faire : transformer une expression \codecml{entree} en \LaTeX en un arbre d'expression. L'id\'ee est de faire du pattern matching, mais sur une cha\^ine de caract\`eres.
En effet, on voudrait une application recursive \codecml{parselatex} telle que par exemple\\
\codecml{parselatex "}x\codecml{+}y\codecml{"} retourne \codecml{F(plus,[|parselatex } x \codecml{ ; parselatex} y \codecml{|] )}\\
Pour se faire, pourquoi ne pas r\'eutiliser notre language d\'efini section 2.1, LanguInv. On voudrait par exemple ici reconna\^itre le pattern suivant en LanguInv : \chcar{\%|0;\%+\%|1;\%}.\\\\
L'id\'ee est la suivante : pour chaque pattern \codecml{p} en LanguInv, utiliser une fonction associ\'ee \`a un automate que nous expliciterons plus loin \codecml{automate\_de\_pattern p} qui retournerait un automate qui \'ecrit reconna\^issant \codecml{entree} et ressortant les chaines de caract\`eres dans \codecml{entree} \`a la place des placeholders de notre motif en LanguInv.\\\\
Pour le pattern du plus et pour \chcar{banane+pomme} comme valeur pour \codecml{entree}, l'automate du pattern de plus reconna\^itrait \codecml{entree} et ressortirait un tableau \codecml{[|"banane","pomme"|]} auquel il ne resterait qu'\`a associer \codecml{F(plus,[|parselatex "banane" ; parselatex "pomme"|])}\\
La difficult\'ee r\'eside en la construction d'un tel automate.\\
Un d\'etail : plusieurs pattern pourraient \^etre reconnus, alors on utilisera seulement la sortie de l'automate du premier de ces pattern dans notre liste de pattern. Ils seront ainsi prioris\'es.\\\\

La tache sera d\'ecoup\'ee en deux \'etapes, il faut construire l'automate \`a partir d'un motif en LanguInv, puis appliquer la fonction associ\'ee \`a cet automate construit \`a du latex comme d\'ecrit pr\'ecedement.
Construisons alors les fonctions \codecml{automate\_de\_pattern} et \codecml{parselatex}.

\subsection{\codecml{automate\_de\_pattern}}
Comme dit pr\'ecedemment, \codecml{automate\_de\_pattern} sera la fonction associ\'ee \`a un automate qui ecrit (qui sera appel\'e \codecml{autoMagique}) ressortant l'automate voulu.\\
Par n\'ecessit\'e, nous allons g\'eneraliser le type de sortie d'un automate qui \'ecrit, en introduisant les espaces d'\'ecriture.
\subsubsection{Espace d'\'ecriture}
Quand on a un automate qui \'ecrit, l'espace dans lequel il \'ecrit a besoin de trois choses :
\begin{itemize}
    \item un type \codecml{'a}
    \item une op\'eration appel\'ee concat\'enation de type \codecml{'a -> 'a -> 'a}
    \item un neutre \codecml{e} de type \codecml{'a} tel que $\forall$ \codecml{x} $\in$ \codecml{'a},  \codecml{concat x e = x}
\end{itemize}
Par exemple pour l'automate de \codecml{evaluelatex}, l'espace d'ecriture est celui des mots, que l'on appelera espace d'\'ecriture \codecml{classique}, alors son type est \codecml{char list}, sa concat\'enation est \codecml{fun a b -> a@b} et son neutre est \codecml{[]}.\\\\
Ici nous aurons besoin des espaces d'\'ecriture suivants : \codecml{dansph} (ph pour placeholder) et autoprio.\\
\codecml{dansph} est alors d\'efini ainsi :
\begin{lstlisting}
type type_dansph = (char list) array

let arrayvide n = Array.make n []

let concat_dansph x y =
	let n = (Array.length x) in
	let res = Array.make n [] in
	for i=0 to n-1 do res.(i)<-x.(i)@y.(i);done;
	res

let dansph n = {neutre = arrayvide n; operation = concat_dansph}
\end{lstlisting}

Pour ce qui est de \codecml{autoprio}, il sera bien plus long \`a expliquer, cela constituera alors une prochaine partie.

\subsection{\codecml{autoMagique}}
L'automate retourn\'e par autoMagique doit alors reconna\^itre un \codecml{char list} et son espace d'\'ecriture sera \codecml{dansph}, ainsi dans le tableau qu'il retournera, \`a l'indice $i$ on retrouvera ce qu'il y a dans \`a la place du placeholder $i$.
Pour reconna\^itre par exemple le motif en LanguInv suivant : \chcar{frac{\%|0;\%}{\%|1;\%}}, le premier automate inagin\'e \'etait celui figure 3.
\unefiguretikz{fig3}{Automate naïf}
Mais avec cet automate d\'eterministe, on rencontre vite un probl\`eme.\\
Pour \chcar{frac\{\{aHA\}\}\{bab\}}, l'automate retournera \`a l'\'etat 6 \codecml{[|\chcar{"\{aHA"},[]|]}, et sera bloqu\'e.\\
Pour rem\'edier \`a ce probl\`eme, l'id\'ee est de passer par un automate non d\'eterministe, o\`u \`a l'\'etat 5, pour la lettre \chcar{\}}, il y ait une transition vers l'\'etat 6 et une vers l'\'etat 5. (Figure 6)\\
\unefiguretikz{fig4}{Automate naïf corrig\'e}
Ainsi deux parcours de l'automate seront possibles :
\begin{itemize}
  \item Le parcours d\'ecrit pr\'ec\'edement, qui \'echoue \`a l'\'etape 6
  \item Le parcours qui reste \`a 5 apr\`es avoir reconnu une premi\`ere fois \chcar{\{}
\end{itemize}
Ce dernier parcours reconna\^it bien le mot et retournera le r\'esultat attendu : \codecml{[|\chcar{"aHA"};\chcar{"bab"}|]}\\
Alors c'est maintenant la question de comment construire cet automate qui \'ecrit avec un autre automate qui \'ecrit, l'autoMagique.\\\\
L'id\'ee est la suivante : On va assembler des automates avec des op\'erations de concat\'enation bien sp\'ecifiques. Tout va r\'esider dans la d\'efinition de l'espace d'ecriture d'autoMagique, que l'on appelera \codecml{autoprio}.\\\\

\subsection{L'espace d'\'ecriture \codecml{autoprio}}
\subsubsection {Le type et le neutre}
Dans cet \codecml{autoprio}, on aura deux "briques" principales, et si on concat\`ene une "brique" A \`a une "brique" B, on veut pas la m\^eme op\'eration de concat\'enation qu' en concat\'enant une "brique" B \`a une "brique" A.\\
Alors on aura un type \codecml{typautoprio} en deux sous type (chacun associ\'e \`a une "brique").\\
La premi\`ere "brique" sera appell\'ee \codecml{autobrique} et la deuxi\`eme \codecml{recoph}.\\

{\Large TODO : AJOUTER FIGURE AUTOBRIQUE ET RECOPH}\\

Alors, associ\'e \`a \codecml{autobrique} nous aurons le type \codecml{AutDom} (automate dominant), dans ce type on aura un \codecml{char} de renseign\'e (correspondant \`a un caract\`ere \`a reconnaitre pour aller d'un \'etat $i$ \`a un \'etat $i+1$), en plus d'un automate non d\'eterministe qui \'ecrit.\\
Associ\'e \`a \codecml{recoph}, nous aurons le type \codecml{Aut} (automate), dans ce type nous aurons un \codecml{int} de renseign\'e (correspondant \`a un indice de placeholder), ainsi qu'un automate non d\'eterministe qui \'ecrit
Enfin nous aurons un \codecml{AutoVide}, neutre pour notre concat\'enation.\\\\

\subsubsection{La concat\'enation}
Pour l'op\'eration de concat\'enation de \codecml{a} et \codecml{b} (a,b de type \codecml{typautoprio}), nous allons discerner les 4 cas suivants, utilisants les fonctions \codecml{ajoufin} et \codecml{agraphefin} :
\begin{itemize}
    \item \codecml{a = Aut(autoa)} et \codecml{Aut(b =autob)}, alors \codecml{concat\_autoprio a b} retournera\\ \codecml{a}
    \item \codecml{a = AutDom(la,autoa)} et \codecml{b = Aut(autob)}, alors \codecml{concat\_autoprio a b} retournera\\ \codecml{Aut(agraphefin autoa autob)}
    \item \codecml{a = Aut(autoa)} et \codecml{b = AutDom(lb,autob)}, alors \codecml{concat\_autoprio a b} retournera\\ \codecml{AutDom('\%',ajoufin autoa autob lb)}
    \item \codecml{a = AutDom(la,autoa)} et \codecml{b = AutDom(lb,autob)}, alors \codecml{concat\_autoprio a b} retournera\\ \codecml{AutDom('\%', ajoufin autoa autob lb)}
\\\\
\codecml{ajoufin auta autb l} retournera l'automate o\`u on "ajoute" l'automate \codecml{autb} \`a la fin de l'automate \codecml{auta}.\\
C'est \`a dire que l'on cr\'ee une nouvelle transition de tous les \'etats finaux de l'automate a vers l'\'etat initial de l'automate b, le nouvel \'etat initial est celui de \codecml{auta} et les nouveaux \'etats finaux sont ceux de \codecml{autb}.\\\\
\codecml{agraphefin auta autb l} retournera l'automate o\`u on "agraphe" l'automate \codecml{autb} de mani\`ere \`a ce que l'\'etat final de \codecml{auta} se voit "fusionn\'e" avec l'\'etat initial de \codecml{autb}.\\
C'est \`a dire que les transitions allant vers un \'etat final de \codecml{auta} sont redirig\'ees vers l'\'etat initial de \codecml{autb}. \\
{\Large TODO : AJOUTER FIGURE AUTOMAGIQUE}\\

\subsubsection{le r\'esultat}
Maintenant, avec cet espace d'\'ecriture, si on note d'un \texttildelow l'op\'eration de concat\'enation, alors\\
(\chcar{f},\codecml{autobrique})\texttildelow(\chcar{r},\codecml{autobrique})\texttildelow(\chcar{a},\codecml{autobrique})\texttildelow(\chcar{c},\codecml{autobrique})\texttildelow(\chcar{\{},\codecml{autobrique})\texttildelow \\ (\codecml{recoph 0})\texttildelow(\chcar{\}},\codecml{autobrique})\texttildelow(\chcar{\{},\codecml{autobrique})\texttildelow(\codecml{recoph 1})\texttildelow(\chcar{\}},\codecml{autobrique}) sera \'egal \`a l'automate Figure4.
\end{itemize}

\subsection{\codecml{autoMagique}}
Voil\`a \`a quoi ressemble notre \codecml{autoMagique}.
\unefiguretikz{fig5}{\codecml{autoMagique}}
C'est le m\^eme que Figure2 mais qui \'ecrit \'a la place des lettre reconnues, les autobriques avec la lettre.
Notre fonction \codecml{automate\_parser} est donc maintenant compl\`ete, c'est la fonction associ\'ee \`a l'\codecml{autoMagique}. Rappelons que cette fonction, \`a un motif en LanguInv, associe un automate qui \'ecrit, qui lui, reconna\^it ou non une entr\'ee par rapport au motif en LanguInv, et retourne si oui, retourne les cha\^ines de caract\`eres \`a la place des placeholders.

\subsection{Pour une liste de motifs}
\subsubsection{La fonction unvrai}
Bien s\^ur, nous n'essaierons pas de reconna\^itre qu'un seul motif par rapport \`a notre entr\'ee.\\
L'id\'ee est la suivante : il faut tester le premier motif de notre liste de motifs. Si il est reconnu, alors on utilise le r\'esultat de ce premier motif, sinon, on passe au suivant, et ainsi de suite.\\
Nous utiliserons alors une fonction \codecml{unvrai : 'a list -> ('a -> bool*'b) -> 'b -> bool*'b}\\
\codecml{unvrai l f neut} retournera $f(x)$ pour $x$ le premier \'element de $l$ tel que \codecml{fst (f x)} soit \`a \codecml{true}. Si \codecml{fst (f x)} est \`a \codecml{false} $\forall x \in l$, alors cette fonction renverra \codecml{(false,neut)}.
\subsubsection{Que faire de ces motifs}
On a une liste de pattern, mais on veut faire du pattern-matching, donc \`a chaque pattern, on veut associer un "match", c'est \`a dire renseigner ce que l'on renvoit en fonction du retour de l'automate associ\'e au pattern. Ce "match" sera une fonction prenant en entr\'ee \`a la fois le retour de l'automate associ\'e au motif, mais prendra aussi la fonction \codecml{parse\_latex} pour pouvoir \'eventuellement l'appliquer r\'ecursivement.\\\\
Voil\'a \`a quoi peut ressembler le pattern-match associ\'e \`a l'op\'eration d'addition :\\
\codecml{\chcar{"\%|0;\%+\%|1;\%"},fun retour parse -> F(plus,[|parse retour.(0) ; parse retour.(1)|])}

\subsubsection{\codecml{parse\_latex}}
Maintenant que nous avons tous les \'elements, nous pouvons enfin de nous lancer dans la construction de \codecml{parse\_latex}.\\
Il suffit en fait simplement d'utiliser la fonction \codecml{unvrai}, pour laquelle il nous faudra un neutre, qui pourra \^etre l'expression \codecml{C(0.)}, avec comme fonction \\
\codecml{fun (patt,funpatt)-> let a,b=reco\_pattern patt mot in if a then (a,funpatt b) else (false, C(0.))}.
On appliquera tout ceci \`a notre liste de pattern-matching.
\section{Une extension : simplifier les expressions}
Il s'agit ici de faire un pattern-matching sur les arbres d'expression. Alors pour nous servir de motif, on utilisera un arbre d'expression, o\`u la variable indic\'ee par $i$ sera le placeholder $i$.\\\\
{\Large TODO : afficher un exemple de ce pattern-matching.}\\
Un exemple de couple pattern-matching, serait l suivant : \\
\codecml{(F(inv,[|F(inv,[| V(0) |])|]),fun simpl retour -> simpl retour.(0))}\\\\
Mais ainsi on aura pas totalement simplifi\'e notre expression, donc une derni\`ere id\'ee c'est de faire ce pattern-matching de notre liste de pattern-match sur notre expression tant qu'une simplification est effectu\'ee.
\listoffigures
\end{document}

%TODO : changer dans automate.ml rendv en rendph
